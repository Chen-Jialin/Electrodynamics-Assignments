% !TEX program = pdflatex
\documentclass[12pt]{article}
 \usepackage[margin=1in]{geometry} 
\usepackage{amsmath,amsthm,amssymb,amsfonts, enumitem, fancyhdr, color, comment, graphicx, environ}
\pagestyle{fancy}
\setlength{\headheight}{65pt}
\newenvironment{problem}[2][Problem]{\begin{trivlist}
\item[\hskip \labelsep {\bfseries #1}\hskip \labelsep {\bfseries #2.}]}{\end{trivlist}}
\newenvironment{sol}
    {\emph{Solution:}
    }
    {
    \qed
    }
\specialcomment{com}{ \color{blue} \textbf{Comment:} }{\color{black}}
\NewEnviron{probscore}{\marginpar{ \color{blue} \tiny Problem Score: \BODY \color{black} }}
\usepackage[UTF8]{ctex}
\usepackage{mathrsfs}
\usepackage{bm}
\lhead{Name: 陈稼霖\\ StudentID: 45875852}
\rhead{PHYS1304 \\ Electrodynamics \\ Spring 2019 \\ Homework 5}
\begin{document}
\begin{problem}{1}
Following the Example 2 above, i.e., a charge $q$ is located at $(d,0,0)$, outside a grounded ball of radius $\pi$ centered at $(0,0,0)$.\\
(a) Find the work done to remove the charge to infinity.\\
(b) Repeat the calculation of the work done to remove the charge to infinity against the force of an isolated charged conducting ball with charge $Q$.
\end{problem}
\begin{sol}
\\(a) As we have proved in Example 2, the force on the charge, $q$, is
\[
F=\frac{q^2R}{4\pi\epsilon_0}\frac{x}{x^4-2R^2x^2+R^4}
\]
when it is place at $(x,0,0)$ ($x>R$).\\
So the work done to remove the charge to infinity is
\[
W=\int_d^{+\infty}Fdx=\frac{q^2R}{8\pi\epsilon_0(d^2-R^2)}
\]
(b) As we have proved in Example 2, the force on the charge, $q$, is
\[
F_1=\frac{q}{4\pi\epsilon_0}(\frac{Q}{x^2}-\frac{qR}{x^3})-F
\]
when it is place at $(x,0,0)$ ($x>R$) near an isolated conducting ball.\\
So the work done to remove the charge to infinity is
\[
W_1=\int_d^{+\infty}F_1dx=\frac{q}{4\pi\epsilon_0}(\frac{1}{d}-\frac{qR}{2d^2}-\frac{qR}{d^2-R^2})
\]
\end{sol}

\begin{problem}{2}
There are two large parallel plates that are grounded. There is a charge $q$ placed between them. The distance to the first and the second plates are $d_1$ and $d_2$, respectively. Find the induced charge $Q_1$ and $Q_2$ and the plates.\\
Solve the example above using the image charge method. (Hint: First consider the case of a wedge of vacuum with conducting boundaries, with the angle of $\pi/n$, and a charge $q$ placed in it. Then take the limit of $\rho,n\to\infty$, where $\rho$ is the distance between the charge and the edge.)
\end{problem}
\begin{sol}

\end{sol}

\begin{problem}{3}
a) Show that the Green function $G(x,x',y,y')$ appropriate for Dirichlet boundary conditions for a square two-dimensional region, $0\leq x\leq1$, $0\leq y\leq1$, has an expression:
\[
G(x,x'y,y')=2\sum_{n=1}^{\infty}g_n\sin(n\pi x)\sin(n\pi x')
\]
where $g_n(y,y')$ satisfies
\[
(\frac{\partial^2}{\partial y'^2}-n^2\pi^2)g_n(y,y')=-4\pi\delta(y-y')
\]
and $g_n(y,0)=g_n(y,1)=0$.\\
b) Taking for $g_n(y,y')$ appropriate linear combinations of $\sinh(n\pi y')$ and $\cosh(n\pi y')$ in the two regions, $y'<y$ and $y'>y$, in accord with the boundary conditions and the discontinuity in slope required by the source delta function, show that the explicit form of $G$ is
\[
G(x,x',y,y')=8\sum_{n=1}\frac{1}{n\sinh(n\pi)}\sin(n\pi x)\sin(n\pi x')\sinh(n\pi y_<)\sinh[n\pi(1-y_>)]
\]
where $y_<(y_>)$ is the smaller (larger) of $y$ and $y'$.
\end{problem}
\begin{sol}

\end{sol}

\begin{problem}{4}
A two-dimensional potential exists on a unit square area ($0\leq x\leq1$, $0\leq y\leq1$) bounded by “surfaces” held at zero potential. Over the entire square there is a uniform charge density of unit strength (per unit length in $z$). Using the Green function of the previous problem, show that
\[
\phi(x,y)=\frac{4}{\pi^3\epsilon_0}\sum_{m=0}^{\infty}\frac{\sin[(2m+1)\pi x]}{(2m+1)^3}\{1-\frac{\cosh[(2m+1)\pi(y-1/2)]}{\cosh[(2m+1)\pi/2]}\}
\]
\end{problem}
\begin{sol}

\end{sol}

\begin{problem}{5}
Find the electric quadrupole moment of the problem above. One more problem is problem 5 on page 71 of the textbook.\\
空心导体球壳的内外半径为$R_1$和$R_2$,球中心置一偶极子$\bm{p}$,球壳上带电$Q$,求空间各点电势和电荷分布。
\end{problem}
\begin{sol}
The electric quadrupole moment of the first problem is
\[
\mathscr{D}_{33}=\frac{l}{2}\frac{l}{2}Q-\frac{l}{2}\frac{l}{2}Q=0
\]
\end{sol}
\end{document}